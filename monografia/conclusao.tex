\chapter{Conclusion}
\label{chap:conclusion}


ICU mortality prediction is a domain-specific problem. Thus, a prediction model learned from a sub-population of patients is likely to fail when tested against data from other population. We investigated this problem by considering four sub-populations of patients that were admitted to different ICU domains. We showed that patients within a specific ICU domain are epidemiologicaly and physiologically different from patients within other domains. Nevertheless, patients across ICU domains still share basic characteristics. This motivates us to propose improved mortality prediction models based on domain adaptation. Specifically, our models learn domain invariant representations from time series ICU data while transferring the complex temporal latent dependencies between ICU sub-populations. The proposed models employ spatial and temporal feature extractors, being thus able to perform dynamic predictions during the ICU stay, potentialy leading to earlier diagnosis and a more effective therapy. Finally, our models produce a mortality risk space, and the dynamics associated with patient trajectories are meaningful and can be very discriminative, enabling clinicians to track risky trends and to gain more insight into their treatment decisions or interventions. Our models provide impressive gains (4\% to 8\%) for early predictions, i.e., predictions within the first $5-20$ hour period after admission. Significant gains (2\% to 4\%) are also observed for predictions performed based on information acquired during the first 48 hours after admission.