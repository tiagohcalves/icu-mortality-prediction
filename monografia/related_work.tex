\chapter{Related Work}
\label{chap:related_work}

In this chapter we bring some of the most relevant researches that guided our work, exposing the methodology used by the authors and how it is correlated to ours. Research on predicting ICU mortality is of great academic interest in medicine~\citep{cai,sun,wu} and in clinical machine learning~\citep{szolovits,decision,luo,nori}, since a good model can help doctors to save lives. A number of researchers have investigated how to correlate ICU data with patient outcomes. In one of the first studies~\citep{ai}, a group of computer scientists, chemists, geneticists, and philosophers of science was brought together to develop a model that could identify parameters in patient data that correlate with its outcome.

\section{Mortality Prediction}

The PhysioNet ICU Mortality Challenge 2012~\citep{silva} provided benchmark data that incorporate evolving clinical data for ICU mortality prediction. As~\citet{revisited} reported, this benchmark data fostered the development of new approaches, leading to up to 170\% improvement over traditional risk scoring systems that do not incorporate such clinical data currently used in ICUs~\citep{saps}. In what follows, we discuss previous work in contrast with ours.

Most of the current work uses the PhysioNet ICU Mortality Challenge 2012 data. The most effective approaches are based on learning discriminative classifiers for specific sub-populations. 

\citet{physionet} proposed a robust SVM classifier, 

while \citet{bera,hamilton} proposed a logistic regression classifier. 

\citet{vairavan} also employed logistic regression classifiers, but coupled them with Hidden Markov Models in order to model time-series data. 

Shallow neural networks were evaluated in~\citep{xia}, 

while a tree-based Bayesian ensemble classifier was evaluated in~\citep{bayes}. 

\citet{krajnak} employed fuzzy rule-based systems for mortality prediction, 

and~\citet{mcmillan} proposed an approach that identifies and integrates information in motifs that are statistically over- or under-represented in ICU time series of patients. 

More recently,~\citet{hyun} proposed a Markov model that accumulates mortality probabilities. Likewise,~\citet{time-series} proposed an approach that models the mortality probability as a latent state that evolves over time. \citet{kdd2} proposed an approach to address the problem of small data using transfer learning in the context of developing risk models for cardiac surgeries. They explored ways to build surgery-specific and hospital-specific models using information from other kinds of surgeries and other hospitals. Their approach is based on weighting examples according to their similarity to the target task training examples. The three aforementioned works are considered as baselines and compared with our approach.

Following~\citet{kdd2}, in this work we use feature transference, but in a quite different way, as follows: (i) instead of applying instance weighting, we employed a deep model that transfers domain-shared features; (ii) we studied a broader scenario that includes diverse ICU domains; and (iii) our models employ temporal feature extraction, being able to predict patient outcomes dynamically.

\section{ICU Domains and Sub-Populations}

Imbalanced data~\citep{imbalance}, sub-populations of patients with different marginal distributions over their feature spaces~\citep{nori}, and sparse data acquired from heterogeneous sources~\citep{szolovits2,het} are issues that pose significant challenges for ICU mortality prediction. 

\citet{gong} discussed problems due to the lack of consistency in how semantically equivalent information is encoded in different ICU databases. \citet{imbalance} discussed the problem of imbalanced ICU data, which occurs when one of the possible patient outcomes is significantly under-represented in the data. Further, since features are often imbalanced, some ICU domains have a significantly larger number of observations than others (e.g., respiratory failure in adults vs. children). In a recent work,~\citet{icde} proposed a mortality study based on the notion of burstiness, where high values of burstiness in time-series ICU data may relate to possible complications in the patient's medical condition and hence provide indications on the mortality.

While most studies on mortality prediction for ICU patients have assumed that one common risk model could be developed and applied to all the patients,~\citet{nori} advocated that this might fail to capture the diversity of ICU patients. As shown by~\citet{lifetime}, as well as by~\citet{longi}, models built using patient data from particular age groups perform poorly on other age groups because the features used to train the models have different distributions across the groups.

\section{Distribution-Aware Neural Network}


\section{Present Work}

None of the aforementioned approaches attempted to perform ICU domain adaptation, which is a core focus of our work. There is often a mismatch between different ICU domains or patient sub-populations, and domain adaptation seems to be a natural solution for learning more robust models, as different ICU domains share features that exhibit different distributions. While data in different ICU domains may vary, there are potentially shared or local invariant features that shape patients in different ICU domains.

Other focus of our work is to capture spatial and temporal features from time-series ICU data. Features are captured in a way that the state of the patient in a certain time depends on the previous state. This forms a mortality risk space, and trajectories in this space allow to easily describe the state of the patient at a particular time, helping intensivists to estimate the patient progress from the current patient state.