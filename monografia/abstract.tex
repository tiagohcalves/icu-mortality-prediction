Early recognition of risky trajectories during an Intensive Care Unit (ICU) stay is one of the key steps towards improving patient survival. Learning such trajectories from epidemiological and physiological parameters that are continuously measured during an ICU stay requires learning time-series features that are robust and discriminative across diverse patient populations. Patients within different ICU populations (or domains) may vary by age, conditions and interventions, and models built using patient data from a particular ICU domain perform poorly in other domains because the features used to train such models have different distributions across the groups. In this paper, we propose a deep model to capture and transfer complex spatial and temporal features from multivariate time-series ICU data. Features are captured in a way that the state of the patient in a certain time depends on the previous state. This enables dynamically predictions and creates a mortality risk space, allowing to easily describe the risk of the patient at a particular time.
%Our model learns domain invariant features while transferring complex temporal features between ICU domains, thus enabling model adaptation from multiple patient populations. %making our model robust to heterogeneous patient populations. 
A comprehensive cross-ICU experiment with diverse domains reveals that our model outperforms all considered baselines. Gains in terms of AUC range from 4\% to 8\% for early predictions, when compared with a recent state-of-the-art representative for ICU mortality prediction. Our experiments also show the importance of learning models that are specific for each ICU domain. In particular, models for the Cardiac domain achieve AUC numbers as high as 0.87, showing excellent clinical utility for early mortality prediction.

%Our model remains competitive as more observations are acquired during the stay. Specifically, it still offers improvements of roughly 1\% (0.811 vs. 0.805) for predictions within the first 48 hours after the patient admission.